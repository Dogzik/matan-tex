\documentclass[paper=a4, fontsize=17pt]{article}

\usepackage[russian]{babel}
\usepackage{scrextend}
\usepackage[utf8x]{inputenc}
\usepackage[T1,T2A]{fontenc}
\usepackage[left=1.5cm,right=1.5cm,top=1.5cm,bottom=1.5cm,bindingoffset=0cm]{geometry}
\usepackage[pdftex]{graphicx}
\usepackage{amsmath}
\usepackage{mathtools}
\usepackage{ulem}
\usepackage{mathrsfs}
\usepackage{amsfonts}
\usepackage{dsfont}
\usepackage{amssymb}
\usepackage{cmap}
\usepackage{hyperref}
\usepackage{tikz}

\DeclareMathOperator*{\esssup}{ess\, sup}


\parindent=0cm

\title{Определения по матану, семестр 4}

\begin{document}
	\maketitle
	\tableofcontents
	\newpage

	\section{Свойство, выполняющееся почти везде}
	$<X,\mathds{A},\mu>$  - пространство с мерой, и $\omega (x)$  -- утверждение, зависящее от точки $x$.

	$E := \{x: \omega(x) $ --- ложно\} и $\mu E$ = 0. Тогда говорят, что $\omega (x)$ верно при почти всех (п.в.) $x$.

	\section{Сходимость почти везде}
	$<X,\mathds{A},\mu>$  - пространство с мерой, и $f_n, f: X \rightarrow \overline{\mathds{R}}.$

	Говорим, что $f_n \rightarrow f(x)$ почти везде, если $\{x: f_n(x) \not \rightarrow f(x)\}$ измеримо и имеет меру $0$.

	\section{Сходимость по мере}
	$<X, \mathds{A}, \mu>$ - пространство с мерой \\
	$f_n , f : X \rightarrow \overline{\mathds{R}}$ - п.в. конечны, измеримы\\
	Говорят, что $f_n$ сходится к $f$ по мере $\mu$ (при $n \rightarrow +\infty$) (обозначается $f_n\stackrel{\mu}{\Rightarrow}f$) если $\forall\epsilon > 0$ $\mu(X(|f_n - f| > \epsilon))\stackrel{n\rightarrow+\infty}{\rightarrow} 0$

	\section{Теорема Егорова о сходимости почти везде и почти равномерной сходимости}
	$<X, \mathds{A}, \mu>$ - пространство с мерой, $\mu(X) < + \infty$\\
	$f_n , f : X \rightarrow \mathds{R}$ - п.в. конечны, измеримы \\
	$f_n \rightarrow f$ почти всюду. \\
	Тогда $\forall \epsilon > 0\ \exists X_{\epsilon} \subset X, \mu(X \setminus X_{\epsilon}) < \epsilon$, $f_n$ равномерно сходится к $f$ на $X_{\epsilon}$

	\section{Интеграл ступенчатой функции}
	$<X, \mathds{A}, \mu>$ - пространство с мерой.

	$f = \sum\limits_{k = 1}^{n}(\lambda_k \cdot \chi_{E_k})$ - ступенчатая функция, $E_k$ - измеримые дизъюнктные множества, $f \geqslant 0$.

	Интегралом ступенчатой функции $f$ на множестве $X$ назовём $$\int\limits_X f d\mu := \sum\limits_{k = 1}^{n} \lambda_k \cdot \mu E_k$$

	Будем считать, что $[0 \cdot \infty = 0]$.

	\section{Интеграл неотрицательной измеримой функции}
	$<X, \mathds{A}, \mu>$ - пространство с мерой.

	$f$ - измеримо, $f \geqslant 0$, её интегралом на множестве $X$ назовём

	$$\int\limits_{X} f d\mu := sup (\int\limits_X g d\mu)$$

	по всем $g$: $0 \leqslant g \leqslant f, g - $ступенчатая.

	\section{Суммируемая функция}
	$<X, \mathds{A}, \mu>$ - пространство с мерой.

	$f - $измерима, $\int\limits_{X}f^+$ или $\int\limits_{X}f^-$ конечен (хотя бы один из них).

	Тогда интегралом $f$ на $X$ назовём $$\int\limits_{X}f d\mu := \int\limits_{X}f^+ - \int\limits_{X}f^-$$

	Если конечен $\int\limits_{X} f$ (то есть конечны интегралы по обеим срезкам), то $f$ называют суммируемой.

	\section{Интеграл суммируемой функции}
	$<X, \mathds{A}, \mu>$ - пространство с мерой.

	$f -$ измерима, $E \in \mathds{A}$.

	Тогда интегралом $f$ на множестве $E$ назовём

	$$\int\limits_{\mathds{E}}f d\mu := \int\limits_{X}f \cdot \chi(E) d\mu$$

	$f$ суммируемая на $E$, если $\int\limits_{X}f^+ \chi(E)$ и $\int\limits_{X}f^- \chi(E)$ конечны.

	\section{Произведение мер}
	$<X, \mathds{A}, \mu>$, $<Y, \mathds{B}, \nu>$ - пространства с мерой.\\
	$\mu$, $\nu$ - $\sigma$-конечные меры.\\
	$\mathds{A} \times \mathds{B} = \{A\times B \subset X \times Y : A \in \mathds{A}, B \in \mathds{B} \}$ \\
	$m_0 : \mathds{A} \times \mathds{B} \rightarrow \overline{\mathds{R}}$\\ $m_0(A \times B) = \mu A \cdot \nu B$

	$m$ - называется произведением мер $\mu$ и $\nu$, если $m$ - мера, которая ялвяется Лебеговским продолжением $m_0$ с полукольца $\mathds{A} \times \mathds{B}$ на некоторую $\sigma$-алгебру $\mathds{A} \otimes \mathds{B}$.\\
	$m = \mu \times \nu$ - обозначение. \\
	$<X \times Y, \mathds{A} \otimes \mathds{B}, \mu \times \nu>$ - произведение пространств с мерой.

	\section{Теорема Фубини}
	$<X, \mathds{A}, \mu>$, $<Y, \mathds{B}, \nu>$ - пространства с мерой,

	$\mu$, $\nu$ --- $\sigma$-конечные и полные,

	$m = \mu \times \nu$,

	$f$ --- суммируемая на $X \times Y$ по $m$.

	\textbf{Тогда:}
	\begin{itemize}
		\item
		при почти всех $x$ функция $f_x \in L(Y,\nu)$, то есть суммируема на $Y$ по $\nu$

		при почти всех $y$ функция $f^y \in L(X,\mu)$

		\item
		$$x \mapsto \phi(x) \mid \phi(x) = \int\limits_{Y}f_x d\nu \in L(X,\mu)$$

		$$y \mapsto \psi(y) \mid \psi(y) = \int\limits_{X}f^y d\mu \in L(Y,\nu)$$

		Эти функции суммируемы (по $\mu$ в $X$ и по $\nu$ в $Y$ соответствено).

		\item
		$$\int\limits_{X \times Y} f dm
		= \int\limits_{X}\phi(x) d\mu
		= \int\limits_{X} (\int\limits_{Y} f d\nu) d\mu$$

		$$\int\limits_{X \times Y} f dm
		= \int\limits_{Y}\psi(y) d\nu
		= \int\limits_{Y} (\int\limits_{X} f d\mu) d\nu$$
	\end{itemize}

	\section{Образ меры при отображении}
	$<X, \mathds{A}, \mu>$~--- пространство с мерой, $<Y, \mathds{B}, \_>$~--- пространство с $\sigma$-алгеброй.

	$\Phi: X \to Y$, $\Phi^{-1}(\mathds{B}) \subset \mathds{A}$ (прообраз любого множества из $\mathds{B}$ лежит в $\mathds{A}$).

	Пусть для $\forall E \in \mathds{B}$ $\nu(E) = \mu(\Phi^{-1}(E))$.

	$\nu$ является мерой на $Y$ и называется образом меры $\mu$ при отображении $\Phi$.

	\section{Взвешенный образ меры}
	$<X, \mathds{A}, \mu>$~--- пространство с мерой, $<Y, \mathds{B}, \_>$~--- пространство с $\sigma$-алгеброй.

	$\Phi: X \to Y$, $\Phi^{-1}(\mathds{B}) \subset \mathds{A}$ (прообраз любого множества из $\mathds{B}$ лежит в $\mathds{A}$).

	$\omega: X \to \overline{\mathds{R}}$, $\omega \geq 0$~--- измеримая.

	Пусть для $E \in \mathds{B}$ $\nu(E) = \int\limits_{\Phi^{-1}(E)} \omega~d\mu$.

	$\nu$ является мерой на $Y$ и называется взвешенным образом меры $\mu$.

	При $\omega \equiv 1$ взвешенный образ меры является обычным образом меры.

	\section{Плотность одной меры по отношению к другой}
	$<X, \mathds{A}, \mu>$~--- пространство с мерой.

	$\omega: X \to \overline{\mathds{R}}$, $\omega \geq 0$~--- измеримая.

	$\nu(E) = \int_E \omega(x)~d\mu$. $\nu$~--- мера на $X$.

	$\omega$ называется плотностью $\nu$ относительно $\mu$.

	\section{Заряд, множество положительности}
	\subsection{Заряд}
	$<X, \mathds{A}, \_>$~--- пространство с $\sigma$-алгеброй.

	$\phi: \mathds{A} \to \mathds{R}$ (конечная, не обязательно неотрицательная).

	$\phi$ счётно аддитивна.

	Тогда $\phi$~--- заряд.

	\subsection{Множество положительности}
	$A \subset X$~--- множество положительности, если
	$\forall B \subset A$, $B$ измеримо: $\phi(B) \geq 0$.

	\section{Сферические координаты в $ R^3 $ и в $ R^m $, их Якобианы}
	$x_1 = r \cdot \cos \phi_1$
    \hfill
	$1 \leq i \leq m-2: \phi_i \in [0,\pi]$

	$x_2 = r \cdot \sin \phi_1 \cdot \cos \phi_2$
    \hfill
	$i=m-1: \phi_i \in [0,2\pi]$

	$x_3 = r \cdot \sin \phi_1 \cdot \sin \phi_2 \cdot \cos \phi_3$
    \hfill
    $r \geq 0$

    $\vdots$\\
	$x_{m-2} = r \cdot \sin \phi_1 \cdot \sin \phi_2 \cdots \sin \phi_{n-3} \cdot \cos \phi_{n-2}$

	$x_{m-1} = r \cdot \sin \phi_1 \cdot \sin \phi_2 \cdots \sin \phi_{n-2} \cdot \cos \phi_{n-1}$


	$x_{m} = r \cdot \sin \phi_1 \cdot \sin \phi_2 \cdots \sin \phi_{n-2} \cdot \sin \phi_{n-1}$\\

	$\mathcal{J} = r^{n-1} \cdot (\sin \phi_1)^{n-2} \cdot (\sin \phi_2)^{n-3} \cdots (\sin \phi_{n-2})^{1} \cdot (\sin \phi_{n-1})^{0}$

	Что тут происходит идейно. Сначала мы проецируем наш $m$-мерный вектор на нормаль к $(m-1)$-мерной гиперплоскости. Потом рассматриваем проекцию на эту гиперплоскость и в ней рекурсивно повторяем процедуру, пока не дойдём до нашего любимого $\mathds{R}^2$. Уже в нём рассматривем обычные полярные координаты (отсюда и другие ограничения на размер угла).

	\section{Интегральные неравества Гельдера и Минковского}
	$<X, \mathds{A}, \mu>$; $f, g : E \subset X \rightarrow \mathds{C}$ ($E$ - изм.)~--- заданы п.в, измеримы.\\
	\subsection{Нераветсво Гельдера}
	$p, q > 1 : \frac{1}{p} + \frac{1}{q} = 1$.
	\emph{Тогда:}
	${\displaystyle \int\limits_E |fg|d\mu \leq \left(\int\limits_E |f|^p d\mu\right)^\frac{1}{p} \cdot \left(\int\limits_E |g|^q d\mu)\right)^\frac{1}{q}}$
	\subsection{Нераверство Минковского}
	$1 \leq p < +\infty$.
	\emph{Тогда:}
	${\displaystyle \left(\int\limits_E |f + g|^p d\mu \right)^\frac{1}{p}
		\leq \left(\int\limits_E |f|^p d\mu\right)^\frac{1}{p}
		+ \left(\int\limits_E |g|^p d\mu\right)^\frac{1}{p}}$

	\section{Интеграл комплекснозначных функции}
	$(X, \mathds{A}, \mu)$ - пространство с мерой. $E \in \mathds{A}$
	
	$f: E \rightarrow \mathds{C}$
	
	$f$ измерима (суммируема), если $Im(f)$ и $Re(f)$ измеримы (суммируема)
	
	$\int_E f =\int_E Re(f) + i \cdot \int_E Im(f) $ 

	\section{Пространство $L_p(E,\mu),\ 1 \leq p < +\infty$}
	$<X, \mathds{A}, \mu>$, $E \in \mathds{A}$.\\
	$L_p'(E, \mu) = \{ f : \text{п.в.}\ E \rightarrow \mathbb{C},\ \text{изм.},\ \int\limits_E |f|^p d\mu < +\infty \}$\\
	Это линейное пространство (по нер-ву Минковского и линейности пространства измеримых функций).\\
	У этого пространства есть дефект~--- если определить норму как $||f|| = \left(\int\limits_E |f|^p\right)^\frac{1}{p}$, то будет сразу много нулей пространства (ненулевые функции, которые п.в. равны $0$, будут иметь норму $0$).
	Поэтому перейдем к фактор-множеству функций по отношению эквивалентности:\\
	$f \sim g$, если $f = g$ п.в.\\
	$ L_p(E, \mu) := L_p'(E, \mu) / \sim$ - лин. норм. пр-во с нормой $||f|| = \left(\int\limits_E |f|^p\right)^\frac{1}{p}$.\\

	\emph{NB1}: Его элементы --- классы эквивалентности обычных функций. Будем называть их тоже функциями. Они не умеют вычислять значение в точке (т.к. можно всегда подменить значение на любое другое и получить представителя все того же класса эквивалентности), но зато их можно интегрировать!\\

	\emph{NB2}: также иногда будем обозначать $||f||_p$ за норму $f$ в пространстве $L_p$.

	\section{Пространство $L_{\infty}(E,\mu)$}
	$L_\infty(E, \mu) =\{f : \text{п.в.}\ E \rightarrow \mathbb{C},\ \esssup\limits_E |f| < +\infty \}$\\
	\emph{NB1}: $||f||_\infty = \esssup\limits_E |f|$.\\

	\emph{NB2}: Новый вид нер-ва Гельдера : $||f \cdot g||_1 \leq ||f||_p \cdot ||g||_q$ (причем можно брать $p = +\infty, q = 1$ или наоборот).

	\section{Существенный супремум}
	$<X, \mathds{A}, \mu>$, $E \subset X$~--- изм., $f : \text{п.в.}\ E \rightarrow \overline{\mathbb{R}}$.\\

	\emph{Тогда}: $\esssup\limits_{x \in E} f(x) = \inf \{A \in R : f(x) \leq A\ \text{при п.в. $x$}\}$.

	В этом определении $A$ - существенная верхняя граница.

	\emph{Свойства}:
	\begin{enumerate}
		\item
		$\esssup\limits_E f \leq \sup\limits_E f$

		\item
		$f(x) \leq \esssup\limits_E f$ при п.в. $x \in E$.

		\item
		$\int\limits_E |fg|d\mu \leq \esssup\limits_E |g| \cdot \int\limits_E |f|d\mu$.
	\end{enumerate}

	\section{Фундаментальная последовательность, полное пространство}
	\subsection{Фундаментальная последовательность}
	$\{a_n\}$ - фунд. посл. в метрическом пр-ве $(X, \rho)$, если $\forall \epsilon > 0 \exists N: \forall n, k > N: \rho(a_n, a_k) < \epsilon$

	\subsection{Полное пространство}
	$X$ - полное пространство, если любая фундаментальная последовательность в нём сходится.

	\section{Плотное множество}
	Множество $A$ плотно во множестве $B$, если $\forall b \in B \ \forall \epsilon > 0$ верно, что $U_\epsilon(b) \cap A \neq \emptyset$.

	\section{Финитная функция}
	$\varphi : \mathbb{R}^m \rightarrow \mathbb{R}$. $\exists$ шар $B: \varphi \equiv 0 $ вне $B$. Тогда $\phi$~--- финитная.
    
    Множество непрерывных финитных функций обозначаем как $C_0(\mathbb{R}^m)$.

	\section{Мера Лебега-Стилтьеса}
	$\mathbb{P}^1$ -- полукольцо ячеек в $\mathds{R}$. \\
	$g : \mathbb{R} \rightarrow \mathbb{R}$ непрерывна, монотонно возрастает. \\
	$\mu [a, b):=g(b) - g(a)$~--- $\sigma$-конечная мера на $\mathbb{P}^1$. \\

	\emph{NB1}: $g$ не обязательно непр., но должна возрастать. \\
	Тогда $g(c \pm 0)=\lim\limits_{x \rightarrow c \pm 0} g(x)$. \\
	$\mu [a, b):=g(b - 0) - g(a - 0)$~--- тоже $\sigma$-конечная мера (если $g$ не непр. слева, то $\mu [a, b):=g(b) - g(a)$~--- не мера (нет непр. слева)). \\


	\emph{Тогда} мерой Лебега-Стилтьеса будем называть меру $\mu_g$, полученную из $\mu$ по теореме о лебеговском продолжении меры.

	\section{Функция распределения}
	$<X, \mathds{A}, \mu> $, $h: X \rightarrow \overline{\mathbb{R}}$~--- измерима, п.в. конечна.\\

	Пусть $\forall t \in \mathbb{R}\quad \mu X(h < t) < +\infty$.\\
	Тогда $H(t):=\mu X(h < t)$~--- это функция распределения функции $h$ по мере $\mu$.

	\section{Измеримое множество на простой двумерной поверхности в $\mathds{R}^3$}
	
	$ M \subset R^3 $ -- простое 2-мерное многообразие, $ C^1 $ гладкости.

	$ \phi : \underset{\text{откр. обл.}}{O} \subset R^2 \rightarrow R^3$, $ \phi \in C^1 $ -- гомофорфизм, $ \phi(O) = M $

	$ E \subset M $ -- изм. по Лебегу, если $ \phi^{-1}(E) $ -- изм. по Лебегу в $ R^2 $
	
	\section{Мера Лебега на простой двумерной поверхности в $\mathds{R}^3$}

	$ S(E) := \iint\limits_{\phi^{-1}(E)} | \phi_u' \times \phi_v'| dudv $ -- взвеш. образ меры Лебега отн. $ \phi $. Значит это мера на $ \mathbb{A}_{M} $

	\section{Поверхностный интеграл первого рода}
        
        $ M $ -- простое, гл, 2-мерное в $ R^3 $, $ \phi $ -- параметризация

	$ f $ -- изм. отн. $S$ (см. выше), $ f > 0 $ (или $ f $ -- суммируем. по $ S $)

        \emph{Тогда}: $ \int_M f dS$ -- называет инт. первого рода функ. $ f $ по поверхности $M$

	\section{Кусочно-гладкая поверхность в $\mathds{R}^3$}
	
        $M \subset \mathbb R^3$ называется кусочно-гладкой, если $M$ представляет собой объединение:

	* конечного числа простых гладких поверхностей

	* конечного числа простых гладких дуг

	* конечного числа точек

	
	\section{Гильбертово пространство}
	$\mathds{H}$~--- линейное пространство над $\mathds{R}$ или $\mathds{C}$, в котором задано скалярное произведение, и полное относительно соответствуйющей нормы, называется Гильбертовым.

	\section{Ортогональный ряд}
	$x_k \in \mathds{H}, \sum x_k$ называется ортогональным рядом, если $\forall k, l: k \neq l: x_k \bot x_l$.

	\section{Сходящийся ряд в Гильбертовом пространстве}
	$x_n \in \mathds{H}$.

	$\sum x_n$ сходится к $x$, если

	$S_n := \sum\limits_{k = 1}^n x_k$, $S_n \rightarrow x$ (то есть, $|S_n - x| \rightarrow 0$~--- сходимость по норме).

	\section{Ортогональное семейство векторов}
	$\{e_k\} \in \mathds{H}$ - ортогональное семейство векторов, если $\forall k \neq l: e_k \bot e_l$, $\forall k: e_k \neq 0$.

	\section{Ортонормированное семейство векторов}
	$\{e_k\} \in \mathds{H}$ - ортонормированное семейство векторов, если ${e_k}$~--- ортогональное семейство векторов, и $\forall k: |e_k| = 1$.

	\section{Коффициенты Фурье}
	$\{e_k\}$ - ортонормированная система в $\mathds{H}, x \in \mathds{H}$.

	$c_k(x) = \dfrac{\langle x, e_k \rangle}{|e_k|^2}$ называются коэффициентами Фурье вектора $x$ по ортогональной системе $\{e_k\}$.

	\section{Ряд Фурье в Гильбертовом пространстве}

	$\sum c_k(x) \cdot e_k$ называется рядом Фурье вектора $x$ по ортогональной системе $\{e_k\}$.

	\section{Базис, полная, замкнутая ОС}

	$\{e_k\}$~--- ортогональная система в $\mathds{H}$.

	\begin{enumerate}

		\item $\{e_k\}$~--- \textbf{базис}, если $\forall x \in \mathds{H}:\ \exists c_k$, что $x = \sum\limits_{k=1}^{+\infty} c_k \cdot e_k$

		\item $\{e_k\}$~--- \textbf{полная} О.С., если $(\forall k: z \bot e_k) \Rightarrow z = 0$.

		\item $\{e_k\}$~--- \textbf{замкнутая} О.С., если $\forall x \in \mathds{H}: \sum\limits_{k=1}^{+\infty} |c_k(x)|^2 \cdot ||e_k||^2 = ||x||^2$.

	\end{enumerate}

	\section{Сторона поверхности}

	Сторона (простой) гладкой двумерной поверхности {{---}} непрерывное поле единичных нормалей. Поверхность, для которой существует сторона, называется двусторонней. Если же стороны не существует, она называется односторонней.


	\section{Задание стороны поверхности с помощью касательных реперов}

	$F_1, F_2$ -- два касательных векторных поля к поверхности $M$.\\
	$\forall p \in M \quad F_1(p), F_2(p)$ -- Л.Н.З. касательные векторы.\\
	Тогда поле нормалей стороны определяется, как $n := F_1 \times F_2$\\

	Реп\'{е}р - пара векторов из $F_1 \times F_2$.

	\section{Интеграл II рода}

	$M$~--- простая гладкая двусторонняя двумерная поверхность в $\mathds{R}^3$.\\
	$n_0$~--- фиксированная сторона (одна из двух).\\
	$F : M \rightarrow \mathbb{R}^3$ -- векторное поле.\\

	\emph{Тогда} интегралом II рода назовем $\int\limits_{M} \langle F, n_0 \rangle ds$

	\emph{Замечания}
	\begin{enumerate}
		\item Смена стороны эквивалентна смене знака.
		\item Не зависит от параметризации.
		\item
		$F=(P, Q, R)$.

		Тогда интеграл имеет вид $\iint P dydz + Q dzdx + R dxdy$.

		\emph{NB:} $Q dxdz = -Q dzdx$.
	\end{enumerate}

	\section{Ориентация контура, согласованная со стороной поверхности}

	Ориентация контура согласована со стороной поверхности, если она задает эту сторону.\\

	\emph{Пояснение}:
	Рассмотрим некоторый контур (замкнутую петлю) и точку на нем. Построим два касательных вектора к контуру в этой точке: первый~--- снаружи от контура (задает направление <<движения>> по петле), второй~--- внутри контура. Тогда будем называть такую ориентацию согласованной со стороной, если направление векторного произведения первого и второго векторов в точке совпадает с направлением нормали к поверхности.

	\begin{center}
		\begin{tikzpicture}

			\draw (2,2) ellipse (3cm and 2cm);
			\draw[thick,->] (5,2) -- (5,3) node[anchor=south] {1};
			\draw[thick,->] (5,2) -- (2.4,2) node[anchor=south] {2};

		\end{tikzpicture}
	\end{center}

	\section{Тригонометрический ряд}

	\begin{itemize}
		\item $$ \frac{a_0}{2}  + \sum_{k = 1}^{\infty} a_k\cos kx + b_k\sin kx $$
		(где $ a_i, b_i $ -- коэффициенты ряда).

		\item Другая форма:	$$ \sum_{k = \mathbb{Z}} c_k e^{ikx} $$

		Тогда $ S_n := \sum_{k = -N}^{N} c_k e^{ikx} $.
	\end{itemize}

	\section{Коэффициенты Фурье функции}

	\begin{itemize}
		\item $$ a_k(f) = \frac{1}{\pi} \int_{-\pi}^{\pi} f(x) \cos kx ~ dx $$

		\item $$ b_k(f) = \frac{1}{\pi} \int_{-\pi}^{\pi} f(x) \sin kx ~ dx $$

		\item $$ с_k(f) = \frac{1}{2\pi} \int_{-\pi}^{\pi} f(x) e^{-ikx} ~ dx $$

	\end{itemize}

	\section{Ядро Дирихле и Фейера}

	\subsection{Ядро Дирихле}

	$$ D_n(t) = \frac{1}{\pi}( \frac{1}{2} + \sum_{k = 1}^{n} \cos kt) $$

	\subsection{Ядро Фейера}

	$$ \Phi_n(t) = \frac{1}{n+1} \sum_{k = 0}^{n} D_k(t) $$

	\section{Ротор, дивергенция векторного поля}
    $F = (P, Q, R)$~--- векторное поле в $\mathds{R}^3$.

	$(P, Q, R) \rightarrow (R'_y - Q'_z, P'_z - R'_x, Q'_x - P'_y)$. Такое преобразование называется ротором или вихрем. Обозначается как $rot\ F$.

	$div\ F = P'_x + Q'_y + R'_z$. Многомерный случай определяется аналогично.

	\section{Соленоидальное векторное поле}
	Векторное поле $A$~--- соленоидальное, если $\exists$ векторное поле $B: rot\ B = A$. Тогда $B$ называется векторным потенциалом поля $A$.

	\section{Бескоординатное определение ротора и дивергенции}
	$rot\ F$~--- это такое векторное поле, что $\forall a \ \forall n_0 (rot F(a))_{n_0} = \lim_{r\to 0} \frac{1}{\pi r^2} \int_{\partial B_r} F_ldl$
\\

	$div F(a) = \lim_{r\to 0} \frac{1}{\lambda_3(B(a,r))} \iiint_{B(a,r)} div F_l \,dx\,dy\,dz = \\ \lim_{r\to 0} \frac{1}{\lambda_3(B(a,r))} \iint_{\partial B(a,r)} <F, n_0>\,dS$

	\section{Свертка}

	$$ (f \ast K)(x) = \int_{-\pi}^{\pi} f(x-t)K(t) dt$$

	где $ f, K \in L_1([-\pi, \pi])$.

\section{Аппроксимативная единица. TODO}
	{\color{red} \textbf{TODO}}
\section{Усиленная аппроксимативная единица. TODO}
	{\color{red} \textbf{TODO}}
\section{Метод суммирования средними арифметическими. TODO}
	{\color{red} \textbf{TODO}}
\section{Суммы Фейера. TODO}
	{\color{red} \textbf{TODO}}
\section{Преобразование Фурье. TODO}
	{\color{red} \textbf{TODO}}
\section{Свертка в $L^1(\mathds{R}^m)$. TODO}
	{\color{red} \textbf{TODO}}
\section{Интеграл Фурье, частичный интеграл Фурье. TODO}
	{\color{red} \textbf{TODO}}


	\section{Несобственный интеграл по мере \Large Deprecated}

	$$ \int_{a}^{\rightarrow b} f d\lambda_1 = \lim\limits_{B \rightarrow b-0} \int_{a}^{B} f d\lambda_1 $$

	где $ f $ - локально суммируемая (т. е. $ \forall [a, B] \subset [a, b) ~ f $~--- сумм. на $ [a, B] $)

	\section{$ L_{loc} $ \Large Deprecated}

	$ f : X \times Y \rightarrow \overline{\mathbb{R}}$

	$ (X, \mathds{A}, \mu) $~--- пространство с мерой.

	$Y$~--- метрическое пространство (или метризуемое).

	$\forall y \ f^y(x) = f(x, y)$~--- суммируема на $X$.

	$f$ удовлетворяет $L_{loc}$ ($ f \in (L_{loc}) $) если:
	\begin{itemize}
		\item $ \exists g : X \rightarrow \overline{\mathbb{R}} $~--- суммируема.
		\item $ \exists U(a) \ \forall y \in \dot{U} (a) \ \text{при п. в.} ~ x \in \mathbb{X} ~ |f(x, y)| \leq g(x)$
	\end{itemize}

\end{document}
