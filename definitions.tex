\documentclass[paper=a4, fontsize=15pt]{article}

\usepackage[russian]{babel}
\usepackage{scrextend}
\usepackage[utf8x]{inputenc}
\usepackage[T1,T2A]{fontenc}
\usepackage[left=1.5cm,right=1.5cm,top=1.5cm,bottom=1.5cm,bindingoffset=0cm]{geometry}
\usepackage[pdftex]{graphicx}
\usepackage{amsmath}
\usepackage{mathtools}
\usepackage{ulem}
\usepackage{mathrsfs}
\usepackage{amsfonts}
\usepackage{dsfont}
\usepackage{amssymb}
\usepackage{cmap}
\usepackage{hyperref}

\DeclareMathOperator*{\esssup}{ess\, sup}


\parindent=0cm

\title{Определения по матану, семестр 4}

\begin{document}
\maketitle
\tableofcontents
\newpage

\section{Свойство, выполняющееся почти везде}
$ (X,\mathds{A},\mu)$  - пространство с мерой, и $\omega (x)$  -- утверждение, зависящее от точки $x$.

$E := \{x: \omega(x) $ --- ложно\} и $\mu E$ = 0. Тогда говорят, что $\omega (x)$ верно при почти всех (п.в.) $x$.

\section{Сходимость почти везде}
$ (X,\mathds{A},\mu)$  - пространство с мерой, и $f_n, f: X \rightarrow \overline{\mathds{R}}.$

Говорим, что $f_n \rightarrow f(x)$ почти везде, если $\{x: f_n(x) \not \rightarrow f(x)\}$ измеримо и имеет меру $0$.

\section{Сходимость по мере}
$(X, a, \mu)$ - пространство с мерой, $\mu \cdot X < +\infty$ \\
$f_n , f : X \rightarrow \overline R$ - п.в. конечны\\
Говорят, что $f_n$ сходится к $f$ по мере $\mu$ (при $n \rightarrow +\infty$) (обозначается $f_n\stackrel{\mu}{\Rightarrow}f$) если $\forall\epsilon > 0$ $\mu(X(|f_n - f| > \epsilon))\stackrel{n\rightarrow+\infty}{\rightarrow} 0$

\section{Теорема Егорова о сходимости почти везде и почти равномерной сходимости}
$(X, a, \mu)$ - пространство с мерой\\
$f_n , f : X \rightarrow R$ - п.в. конечны, измеримы \\
$f_n \rightarrow f$.\\
Тогда эта сходимость ``почти равномерная''

\section{Интеграл ступенчатой функции}
$<\mathds{X}, \mathds{A}, \mu>$ - пространство с мерой

$f = \sum\limits_{k = 1}^{n}(\lambda_k \cdot \chi_{E_k})$ - ступенчатая функция, $E_k$ - измеримые дизъюнктные множества, $f \geqslant 0$

Интегралом ступенчатой функции $f$ на множестве $\mathds{X}$ назовём $$\int\limits_\mathds{X} f d\mu := \sum\limits_{k = 0}^{n} \lambda_k \cdot \mu E_k$$

Будем считать, что $[0 \cdot \infty = 0]$

\section{Интеграл неотрицательной измеримой функции}
$<\mathds{X}, \mathds{A}, \mu>$ - пространство с мерой

$f$ - измеримо, $f \geqslant 0$, её интегралом на множестве $\mathds{X}$ назовём

$$\int\limits_{\mathds{X}} f d\mu := sup (\int\limits_{\mathds{X}} g)$$

, где $0 \leqslant g \leqslant f, g - $ступенчатая

\section{Суммируемая функция}
$<\mathds{X}, \mathds{A}, \mu>$ - пространство с мерой

$f - $измерима, $\int\limits_{\mathds{X}}f^+$ или $\int\limits_{\mathds{X}}f^-$ конечен (хотя бы один из них).

Тогда интегралом $f$ на $\mathds{X}$ назовём $$\int\limits_{\mathds{X}}f d\mu := \int\limits_{\mathds{X}}f^+ - \int\limits_{\mathds{X}}f^+$$

Тогда если конечен $\int\limits_{\mathds{X}} f,$ (то есть конечны интегралы по обеим срезкам), то $f$ называют суммируемой

\section{Интеграл суммируемой функции}
$<\mathds{X}, \mathds{A}, \mu>$ - пространство с мерой

$f -$ измерима, $E \in \mathds{A}$

Тогда интегралом $f$ на множестве $E$ назовём

$$\int\limits_{\mathds{E}}f d\mu := \int\limits_{\mathds{X}}f \cdot \chi(E) d\mu$$

$f$ суммируемая на $E$, если $\int\limits_{\mathds{X}}f^+ \chi(E)$ и $\int\limits_{\mathds{X}}f^- \chi(E)$ конечны

\section{Произведение мер}
$<\mathds{X}, \alpha, \mu>$, $<\mathds{Y}, \beta, \nu>$ - пространства с мерой\\
$\mu$, $\nu$ - $\sigma$-конечные меры\\
$\alpha \times \beta = \{A\times B \subset \mathds{X} \times \mathds{Y} : A \in \alpha, B \in \beta \}$ \\
$m_0 : \alpha \times \beta \rightarrow \overline R$\\ $m_0(A \times B) = \mu A \cdot \nu B$ \\\\
$m$ - называется произведением мер $\mu$ и $\nu$, если $m$ - мера, которая ялвяется Лебеговским продолжением $m_0$ с полукольца $\alpha \times \beta$ на некоторую $\sigma$-алгебру $\alpha \otimes \beta$.\\
$m = \mu \times \nu$ - обозначение \\
$<\mathds{X} \times \mathds{Y}, \alpha \otimes \beta, \mu \times \nu>$ - произведение пространств с мерой

\section{Теорема Фубини}
$<\mathds{X}, \mathds{A}, \mu>$, $<\mathds{Y}, \mathds{B}, \nu>$ - пространство с мерой,

$\mu$, $\nu$ --- $\sigma$-конечные и полные,

$m = \mu \times \nu$,

$f$ --- суммируемая на $\mathds{X} \times \mathds{Y}$ по $m$.

\textbf{Тогда:}
\begin{itemize}
\item
при <<почти всех>> $x$ функция $f_x \in \mathds{L}(\mathds{Y},\nu)$, то есть суммируема на $\mathds{Y}$ по $\nu$

при <<почти всех>> $y$ функция $f^y \in \mathds{L}(\mathds{X},\mu)$

\item
$$x \mapsto \phi(x) \mid \phi(x) = \int\limits_{\mathds{Y}}f_x d\nu \in \mathds{L}(\mathds{X},\mu)$$

$$y \mapsto \psi(y) \mid \psi(y) = \int\limits_{\mathds{X}}f^y d\mu \in \mathds{L}(\mathds{Y},\nu)$$

Это есть эти функции суммируемы в некотором контексте ($\mathds{X},\mu$ и $\mathds{Y},\nu$ соответсвено)

\item
$$\int\limits_{\mathds{X} \times \mathds{Y}} f dm
= \int\limits_{\mathds{X}}\phi(x) d\mu
= \int\limits_{\mathds{X}} (\int\limits_{\mathds{Y}} f d\nu(y)) d\mu(x)$$

$$\int\limits_{\mathds{X} \times \mathds{Y}} f dm
= \int\limits_{\mathds{Y}}\psi(x) d\nu
= \int\limits_{\mathds{Y}} (\int\limits_{\mathds{X}} f d\mu(x)) d\nu(y)$$
\end{itemize}

\section{Образ меры при отображении}
$(X, \mathds{A}, \mu)$~--- пространство с мерой, $(Y, \mathds{B}, \_)$~--- пространство с $\sigma$-алгеброй.

$\Phi: X \to Y$, $\Phi^{-1}(\mathds{B}) \subset \mathds{A}$ (прообраз любого множества из $\mathds{B}$ лежит в $\mathds{A}$).

Пусть для $\forall E \in \mathds{B}$ $\nu(E) = \mu(\Phi^{-1}(E))$.

$\nu$ является мерой на $Y$ и называется образом меры $\mu$ при отображении $\Phi$.

\section{Взвешенный образ меры}
$(X, \mathds{A}, \mu)$~--- пространство с мерой, $(Y, \mathds{B}, \_)$~--- пространство с $\sigma$-алгеброй.

$\Phi: X \to Y$, $\Phi^{-1}(\mathds{B}) \subset \mathds{A}$ (прообраз любого множества из $\mathds{B}$ лежит в $\mathds{A}$).

$\omega: X \to \overline{\mathds{R}}$, $\omega \geq 0$~--- измеримая.

Пусть для $E \in \mathds{B}$ $\nu(E) = \int\limits_{\Phi^{-1}(E)} \omega~d\mu$.

$\nu$ является мерой на $Y$ и называется взвешенным образом меры $\mu$.

При $\omega \equiv 1$ взвешенный образ меры является обычным образом меры.

\section{Плотность одной меры по отношению к другой}
$(X, \mathds{A}, \mu)$~--- пространство с мерой.

$\omega: X \to \overline{\mathds{R}}$, $\omega \geq 0$~--- измеримая.

$\nu(E) = \int_E \omega(x)~d\mu$. $\nu$~--- мера на $X$.

$\omega$ называется плотностью $\nu$ относительно $\mu$.

\section{Заряд, множество положительности}
\subsection{Заряд}
$(X, \mathds{A}, \_)$~--- пространство с $\sigma$-алгеброй.

$\phi: \mathds{A} \to \mathds{R}$ (конечная, не обязательно неотрицательная).

$\phi$ счётно аддитивна.

Тогда $\phi$~--- заряд.

\subsection{Множество положительности}
$A \subset X$~--- множество положительности, если
$\forall B \subset A$, $B$ измеримо: $\phi(B) \geq 0$.e

\section{Сферические координаты в $ R^3 $ и в $ R^m $, их Якобианы}
$x_1 = r \cdot \cos \phi_1$
$\ \ \ \ \ \ \ \ \ \ \ \ \ \ \ \ \ \ \ \ \ \ \ \ \ \ \ \ \ \ \ \ \ \ \ $ 
$1 \leq i \leq m-2: \phi_i \in [0,\pi]$

$x_2 = r \cdot \sin \phi_1 \cdot \cos \phi_2$
$\ \ \ \ \ \ \ \ \ \ \ \ \ \ \ \ \ \ \ \ \ \ \ \ \ \ \ \ \ \ $
$i=m-1: \phi_i \in [0,2\pi]$

$x_3 = r \cdot \sin \phi_1 \cdot \sin \phi_2 \cdot \cos \phi_3$\\
.\\
.\\
.\\
$x_{m-2} = r \cdot \sin \phi_1 \cdot \sin \phi_2 \cdots \sin \phi_{n-3} \cdot \cos \phi_{n-2}$

$x_{m-1} = r \cdot \sin \phi_1 \cdot \sin \phi_2 \cdots \sin \phi_{n-2} \cdot \cos \phi_{n-1}$


$x_{m} = r \cdot \sin \phi_1 \cdot \sin \phi_2 \cdots \sin \phi_{n-2} \cdot \sin \phi_{n-1}$\\

$\mathcal{J} = r^{n-1} \cdot (\sin \phi_1)^{n-2} \cdot (\sin \phi_2)^{n-3} \cdots
\cdot (\sin \phi_{n-2})^{1} \cdot (\sin \phi_{n-1})^{0}$

Что тут происходит идейно. Сначала мы проецируем наш $m$-мерный вектор на нормаль к $(m-1)$-мерной гиперплоскости. Потом рассматриваем проекцию на эту гиперплоскость и в ней рекурсивно повторяем процедуру пока не дойдём до нашего любимого $\mathbb{R}^2$. Уже в нём рассматривем обычные полярные координаты(отсюда и другие ограничения на размер угла).

\section{Интегральные неравества Гельдера и Минковского}
\subsection{Нераветсво Гельдера}
$(X, \mathds{A}, \mu)\ f, g : E \subset X \rightarrow C$ ($E$ - изм.) --- заданы п.в, измеримы\\
$p, q > 1 : \frac{1}{p} + \frac{1}{q} = 1$. 
\emph{Тогда:}
 ${\displaystyle \int\limits_E |fg|d\mu \leq \left(\int\limits_E |f|^p d\mu\right)^\frac{1}{p} \cdot \left(\int\limits_E |g|^q d\mu)\right)^\frac{1}{q}}$
\subsection{Нераверство Минковского}
$(X, \mathds{A}, \mu)\ f, g $ --- заданы п.в, измеримы\\
$1 \leq p < +\infty$. 
\emph{Тогда:}
${\displaystyle \left(\int\limits_E |f + g|^p d\mu \right)^\frac{1}{p} 
\leq \left(\int\limits_E |f|^p d\mu\right)^\frac{1}{p} 
\cdot \left(\int\limits_E |g|^p d\mu\right)^\frac{1}{p}}$

\section{Интеграл комплекснозначныйх функции}
\textbf{TODO: Лев и Вадим}

\section{Пространство $L_p(E,\mu),\ 1 \leq p < +\infty$}
$(X, \mathds{A}, \mu)\, E \subset \mathds{A}$\\
$L_p'(E, \mu) = \{$ f : п.в. $E \rightarrow \mathbb{C}$, изм., ${\displaystyle \int\limits_E |f|^p d\mu < +\infty} \}$\\
Это линейное пространство (по нер-ву Минковского и линейности пространства измеримых функций).\\
У этого пространства есть дефект - если определить норму как $||f|| = \left(\int\limits_E |f|^p\right)^\frac{1}{p}$, то будет сразу много нулей пространства (ненулевые функции, которые п.в. равны 0 будут давать норму 0).
Поэтому перейдем к фактор-множеству функций по отношению эквивалентности:\\
$f \sim g$, если $f = g$ п.в.\\
$ L_p(E, \mu) := L_p'(E, \mu) /_{\sim}$ - лин. норм. пр-во с нормой $||f|| = \left(\int\limits_E |f|^p\right)^\frac{1}{p}$.\\

\emph{NB1}: Его элементы --- классы эквивалентности обычных функций. Будем называть их тоже функциями. Они не умеют вычислять значение в точке (т.к. можно всегда подменить значение на любое другое и получить представителя все того же класса эквивалентности), но зато их можно интегрировать!\\

\emph{NB2}: также иногда будем обозначать $||f||_p$ за норму $f$ в пространстве $L_p$.

\section{Пространство $L_{\infty}(E,\mu)$}
$L_\infty(E, \mu) =\{$ f : п.в. $E \rightarrow \mathbb{C},\ \esssup\limits_E |f| < +\infty \}$\\
\emph{NB1}: $||f||_\infty = \esssup\limits_E |f|$.\\

\emph{NB2}: Новый вид нер-ва Гельдера : $||f \cdot g||_1 \leq ||f||_p \cdot ||g||_q$ (причем можно брать $p = +\infty, q = 1$ или наоборот).

\section{Существенный супремум}
$(X, \mathds{A}, \mu), E \subset X - $ изм., $f : $ п.в. $E \rightarrow \overline{\mathbb{R}}$.\\

\emph{Тогда}: $\esssup\limits_{x \in E} f(x) = \inf \{A \in R : f(x) \leq A$ п.в. $x \}$. 

В этом определении $A$ - существенная верхняя граница. 

\emph{Свойства}:
\begin{enumerate}
	\item
	$\esssup\limits_E f \leq \sup\limits_E f$
	
	\item
	$f(x) \leq \esssup\limits_E f$ при п.в. $x \in E$.
	
	\item
	$\int\limits_E |fg|d\mu \leq \esssup\limits_E |g| \cdot \int\limits_E |f|d\mu$.
\end{enumerate}

\end{document}
