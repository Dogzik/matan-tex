\documentclass[paper=a4, fontsize=15pt]{article}

\usepackage[russian]{babel}
\usepackage{scrextend}
\usepackage[utf8x]{inputenc}
\usepackage[T1,T2A]{fontenc}
\usepackage[left=1.5cm,right=1.5cm,top=1.5cm,bottom=1.5cm,bindingoffset=0cm]{geometry}
\usepackage[pdftex]{graphicx}
\usepackage{amsmath}
\usepackage{mathtools}
\usepackage{ulem}
\usepackage{mathrsfs}
\usepackage{amsfonts}
\usepackage{dsfont}
\usepackage{amssymb}
\usepackage{cmap}
\usepackage{hyperref}


\parindent=0cm

\title{Теоремы по матану, семестр 4}

\begin{document}
\maketitle
\tableofcontents
\newpage

\section{Характеризация измеримых функций с помощью ступенчатых (формулировка). Следствия}
$(X,\mathds{A},\mu)$~--- пространство с мерой.

$f$~--- измеримая функция на $X$, $\forall x\ f(x) \geq 0$. Тогда $\exists$ ступенчатые функции $f_n$, такие что:
\begin{enumerate}
    \item $\forall x$ $0 \leq f_n(x) \leq f_{n+1}(x) \leq f(x)$.
    \item $f_n(x)$ поточечно сходится к $f(x)$.
\end{enumerate}

Следствие $1$:

$f: X \rightarrow \overline {\mathds{R}}$ измеримая. Тогда $\exists$ ступенчатая $f_n: \forall x:  lim f_n(x) = f(x)$ и $|f_n(x)| \leq |f(x)|$.

\emph{Доказательство:}

\begin{enumerate}
	\item Рассмотрим $f = f^+ - f^-. f^+ = max(f, 0), f^- = max(-f, 0)$. Срезки измеримы: $E(f^+  < a) = E(f < a) \cap E(0 < a)$, при этом $f$  и $g \equiv 0$ измеримы ($f^-$ измерима аналогично).
	\item Срезки измеримы и неотрицательны, тогда по теореме существуют ступенчатые функции $f^+_n \rightarrow f^+, f^-_n \rightarrow f^-$. Тогда и $f^+_n - f^-_n$ это ступенчатая функция, при этом по свойству пределов: $f^+_n - f^-_n \rightarrow f^+ - f^- = f$
	
	(почему верно с модулем -- непонятно, если в лоб, то неверно. Спрошу 19.02)
\end{enumerate}

Следствие $2$:

$f, g$ --- измеримые функции. Тогда $fg$ -- измеримая функция. При этом считаем, что $0 \cdot \infty = 0$.

\emph{Доказательство:}
\begin{enumerate}
	\item Рассмотрим $f_n \rightarrow f: |f_n| \leq |f|, g_n \rightarrow g: |g_n| \leq |g|$ из первого следствия. Тогда $f_ng_n \rightarrow fg$

	(и что с того? У нас же другое определение измеримых функций. Спрошу 19.02)
\end{enumerate}

Следствие $3$:

$f, g$ --- измеримые функции. Тогда $f + g$ -- измеримая функция. При этом считаем, что $\forall x$ не может быть, что $f(x) = \pm \infty, g(x) = \mp \infty$

\emph{Доказательство:}

Доказывается как следствие $2$.

\section{Измеримость монотонной функции}

Пусть $E \subset R^m$~--- измеримое по Лебегу, $E' \subset E, \lambda_m (E \setminus E') = 0, f: E \rightarrow \mathds{R}.$ Пусть сужение $f: E' \rightarrow R$ непрерывно. Тогда $f$ измерима на $E$.

\emph{Доказательство:}

\begin{enumerate}
	\item $E(f < a) = E'(f < a) \cup e(f < a), e:=E \setminus E', \lambda_m(e) = 0$. 
	\item $ E'(f < a)$ открыто в $E'$, так как f непрерывна. Поэтому $E' = G \cap E' \Rightarrow$, где $G$ ~--- открытое в $E$ множество.  Значит, $E'(f<a)$ ~--- измеримо по Лебегу, так как оно является борелевским. 
	\item Но и $e(f<a)$ измеримо, так $\lambda_m(e) = 0$, следовательно $E(f < a)$ измеримо как объединение измеримых множеств
\end{enumerate}

Следствие:

$f: <a, b> \rightarrow \mathds{R}$ монотонна. Тогда $f$ измерима.

\emph{Доказательство:}

Множество разрывов монотонной функции НБЧС множество, поэтому можно воспользоваться доказанной теоремой.

\section{Теорема Лебега о сходимости почти везде и сходимости по мере}

\section{Теорема Рисса о сходимости по мере и сходимости почти везде}

\section{Простейшие свойства интеграла Лебега}
\subsection{Для определения (5)}
\begin{enumerate}
	\item $\int\limits_{\mathds{X}}f$ не зависит от представления $f$ как ступенчатой функции, то есть если $f$ реализуется как $f = \sum\limits_{k}(\lambda_k \cdot \chi_{E_k})$ и как $f = \sum\limits_{l}(\alpha_l \cdot \chi_{G_l})$, интегралы по этим функциям равны
	
	\proofname{:}
	
	Выпишем общее разбиение для этих двух разбиений
	
	Пусть $F_{ij} = E_i \cap G_j$
	
	Тогда $f = \sum\limits_{k}(\lambda_k \cdot \chi_{E_k}) = \sum\limits_{l}(\alpha_l \cdot \chi_{G_l}) = \sum\limits_{i, j}(\lambda_i (= \alpha_j) \cdot \chi_{F_{i, j}})$
	
	$\int f = \sum\limits_{i, j}(\lambda_i \cdot \mu F_{i, j}) = \sum\limits_i (\lambda_i \cdot \sum\limits_j (\mu F_{i, j})) = \sum\limits_i (\lambda_i \cdot \mu E_i) = \int f$ для первого разбиения
	
	Аналогично для второго разбиения получаем 
	
	$\int f = \sum\limits_{i, j}(\lambda_i \cdot \mu F_{i, j}) = \sum\limits_j (\alpha_i \cdot \sum\limits_i (\mu F_{i, j})) = \sum\limits_j (\lambda_j \cdot \mu G_i) = \int f$ для второго разбиения, что и требовалось доказать
	
	\item $f, g$ -измеримые ступенчатые функции, $f \leqslant g$, тогда $\int\limits_{\mathds{X}} f \leqslant \int\limits_{\mathds{X}} g$
	
	\proofname{:}
	
	Пусть $f = \sum\limits_{k}(\lambda_k \cdot \chi_{E_k})$, $g = \sum\limits_{l}(\alpha_l \cdot \chi_{G_l})$
	
	Аналогично доказательству предыдущей теоремы, строим общее ступенчатое разбиение
	
	Пусть $F_{ij} = E_i \cap G_j$
	
	Тогда $\int f = \sum\limits_{i, j}(\lambda_i \cdot \mu F_{i, j}) \leqslant \sum\limits_j(\alpha_j \cdot \mu F_{i, j}) = \int g$, что и требовалось доказать
\end{enumerate}

\subsection{Для определения (6)}

\section{Счетная аддитивность интеграла (по множеству)}
Тут будет вторая теорема

\section{Теорема Леви}

\section{Линейность интеграла Лебега}

\end{document}