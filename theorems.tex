\documentclass[paper=a4, fontsize=13.2pt]{article}

\usepackage[russian]{babel}
\usepackage{scrextend}
\usepackage[utf8x]{inputenc}
\usepackage[T1,T2A]{fontenc}
\usepackage[left=1.5cm,right=1.5cm,top=1.5cm,bottom=1.5cm,bindingoffset=0cm]{geometry}
\usepackage[pdftex]{graphicx}
\usepackage{amsmath}
\usepackage{mathtools}
\usepackage{ulem}
\usepackage{mathrsfs}
\usepackage{amsfonts}
\usepackage{dsfont}
\usepackage{amssymb}
\usepackage{cmap}
\usepackage{hyperref}


\parindent=0cm

\title{Теоремы по матану, семестр 4}

\begin{document}
\maketitle
\tableofcontents
\newpage

\section{Характеризация измеримых функций с помощью ступенчатых (формулировка). Следствия}
$(X,\mathds{A},\mu)$~--- пространство с мерой.

$f$~--- измеримая функция на $X$, $\forall x\ f(x) \geq 0$. Тогда $\exists$ ступенчатые функции $f_n$, такие что:
\begin{enumerate}
    \item $\forall x$ $0 \leq f_n(x) \leq f_{n+1}(x) \leq f(x)$.
    \item $f_n(x)$ поточечно сходится к $f(x)$.
\end{enumerate}

Следствие $1$:

$f: X \rightarrow \overline {\mathds{R}}$ измеримая. Тогда $\exists$ ступенчатая $f_n: \forall x:  lim f_n(x) = f(x)$ и $|f_n(x)| \leq |f(x)|$.

\emph{Доказательство:}

\begin{enumerate}
	\item Рассмотрим $f = f^+ - f^-. f^+ = max(f, 0), f^- = max(-f, 0)$. Срезки измеримы: $E(f^+  < a) = E(f < a) \cap E(0 < a)$, при этом $f$  и $g \equiv 0$ измеримы ($f^-$ измерима аналогично).
	\item Срезки измеримы и неотрицательны, тогда по теореме существуют ступенчатые функции $f^+_n \rightarrow f^+, f^-_n \rightarrow f^-$. Тогда и $f^+_n - f^-_n$ это ступенчатая функция, при этом по свойству пределов: $f^+_n - f^-_n \rightarrow f^+ - f^- = f$. Неравенство с модулем верно при правильных эпсилон-неравенствах.
\end{enumerate}

Следствие $2$:

$f, g$ --- измеримые функции. Тогда $fg$ -- измеримая функция. При этом считаем, что $0 \cdot \infty = 0$.

\emph{Доказательство:}
\begin{enumerate}
	\item Рассмотрим $f_n \rightarrow f: |f_n| \leq |f|, g_n \rightarrow g: |g_n| \leq |g|$ из первого следствия. Тогда $f_ng_n \rightarrow fg$ и $fg$ измерима по теореме об измеримости пределов и супремумов (произведение ступенчатых функций -- ступенчатая функция, значит, измеримая)
\end{enumerate}

Следствие $3$:

$f, g$ --- измеримые функции. Тогда $f + g$ -- измеримая функция. При этом считаем, что $\forall x$ не может быть, что $f(x) = \pm \infty, g(x) = \mp \infty$

\emph{Доказательство:}

Доказывается как следствие $2$.

\section{Измеримость монотонной функции}

Пусть $E \subset R^m$~--- измеримое по Лебегу, $E' \subset E, \lambda_m (E \setminus E') = 0, f: E \rightarrow \mathds{R}.$ Пусть сужение $f: E' \rightarrow R$ непрерывно. Тогда $f$ измерима на $E$.

\emph{Доказательство:}

\begin{enumerate}
	\item $E(f < a) = E'(f < a) \cup e(f < a), e:=E \setminus E', \lambda_m(e) = 0$.
	\item $ E'(f < a)$ открыто в $E'$, так как f непрерывна. Поэтому $E' = G \cap E' \Rightarrow$, где $G$ ~--- открытое в $E$ множество.  Значит, $E'(f<a)$ ~--- измеримо по Лебегу, так как оно является борелевским.
	\item Но и $e(f<a)$ измеримо, так $\lambda_m(e) = 0$, следовательно $E(f < a)$ измеримо как объединение измеримых множеств
\end{enumerate}

Следствие:

$f: <a, b> \rightarrow \mathds{R}$ монотонна. Тогда $f$ измерима.

\emph{Доказательство:}

Множество разрывов монотонной функции НБЧС множество, поэтому можно воспользоваться доказанной теоремой.

\section{Теорема Лебега о сходимости почти везде и сходимости по мере}

$(X, a, \mu)$ - пространство с мерой, $\mu \cdot X < +\infty$ \\
$f_n , f : X \rightarrow \overline R$ - п.в. конечны, измеримы \\
$f_n \rightarrow f$ (поточечно, п.в.)

\emph{Доказательство:}
\begin{enumerate}
	\item
	подменим значения $f_n$ и $f$ на некотором множестве меры $0$ так, чтобы сходимость $f_n \rightarrow f$ была всюду.
	(Так можно сделать. Действительно, $f_n \rightarrow f$ на $X \setminus e$, $\mu e = 0$ \\
	$f_n$ - конечно на $X \setminus e_n$,\\
	$f$ - конечно на $X \setminus e_0$.\\
	Тогда на $(X \setminus \bigcup\limits_{n=0}^{+\infty}e_n)$ функции конечны и есть сходимость $f_n \rightarrow f$. По свойствам меры $\mu \bigcup\limits_{n=0}^{+\infty}e_n = 0$. Тогда определим на $\bigcup\limits_{n=0}^{+\infty}e_n$ $f_n = f = 0$. Это очевидно даст нам необходимую конечность и поточечную сходимость.
	)
	\item (частный случай)
	$f_n \rightarrow f \equiv 0$. Тогда пусть $\forall x f_{n}(x)$ - монотонно (по $n$). $|f_{n}(x)|$ - убывает с ростом $n$ и $X(|f_{n}| \geq \epsilon) \supset X(|f_{n+1}| \geq \epsilon)$. А также $\bigcap\limits_{n=0}^{+\infty}X(|f_{n}|\geq\epsilon) = \emptyset$.\\
	$$\begin{cases}
   		$$\mu X < +\infty $$\\
   		$$\ldots\supset E_{n}\supset E_{n+1}\supset\ldots $$
 	\end{cases}$$ $ \Rightarrow \mu E_{n}\rightarrow\mu\cup E_{n}$ - Th о непрерывности меры сверху.\\
 	$\Rightarrow\mu X(|f_{n}\geq\epsilon|) \rightarrow \mu\emptyset = 0$
 	\item (общий случай)
 	$f_n \rightarrow f$. Рассмотрим $\phi_{n}(x) := \sup_{k\geq n}|f_{k}(x) - f(x)|$. Заметим свойства $\phi$:\\
 	$$\begin{cases}
   		$$\phi_{n}(x) \rightarrow 0 $$\\
   		$$\phi_{n} \downarrow_n $$
 	\end{cases}$$
 	$X(|f_{n} - f|\geq\epsilon) \subset X(|\phi_{n}\geq\epsilon|) \Rightarrow $ по монотонности меры имеем $\mu X(|f_{n} - f|\geq\epsilon) \leq \mu X(\phi_{n}\geq\epsilon) \stackrel{part.case}{\longrightarrow} 0$, ч.т.д.
\end{enumerate}


\section{Теорема Рисса о сходимости по мере и сходимости почти везде}

$(X, a, \mu)$ - пространство с мерой\\
$f_n , f : X \rightarrow R$ - п.в. конечны, измеримы \\
$f_n  \stackrel{\mu}{\Rightarrow} f$.\\
Тогда $\exists n_{k}\uparrow $ : $f_{n_{k}} \rightarrow f$ п.в.

\emph{Доказательство:}
$\forall k ~ \mu X(|f_n - f| \geq \frac{1}{k}) \stackrel{n\rightarrow+\infty}{\rightarrow} 0$\\
Тогда $\exists n_{k} : \forall n \geq n_{k} \mu X(|f_n - f| \geq \frac{1}{k}) < \frac{1}{2k}$ (можно считать $n_1 < n_2 < \ldots$)\\
Проверим $f_{n_k} \rightarrow f$ п.в. :
	$E_k := \bigcap\limits{j=k}^{+\infty}X(|f_{n_j} - f|\geq\frac{1}{j})$\\
	$E_1 \supset E_2 \supset E_3 \supset \ldots$\\
	$E_0 := \bigcap\limits{k\in N}E_k$.\\
	$\mu E_k \geq \sum_{j=k}^{+\infty}\mu X(|f_{n_j}-f|\geq\frac{1}{j}) \geq \sum_{j=k}^{+\infty}\frac{1}{2^j} = \frac{1}{2^(k-1)}$ - конечно $\Rightarrow \mu E_k \rightarrow \mu E_0 \Rightarrow \mu E_0 = 0$ (т.к. $\mu E_k \rightarrow 0$).\\
	Рассмотрим $X\not\in E_0$, т.е. если $X\not\in E_0$, то $\exists k : X\not\in E_k$, тогда $\forall j\geq k |f_n(x) - f(x)| < \frac{1}{j}$ при $n \geq n_j$, т.е. $f_{n_k} \rightarrow f$, ч.т.д.
\emph{Следствие:}
$f_n \Rightarrow f $ $ |f_n| \leq g$ п.в.
\emph{Док-во:}
Рассмотрим последовательность $f_{n_k}$ где $f_{n_k} \rightarrow f$ п.в. и вдоль нее применим Th о двух городовых.
$$\begin{cases}
   	$$f_{n_k}(x) \rightarrow f(x)  \forall x \in X\setminus e_1$$\\
   	$$|f_n(x)| \leq g(x)  \forall x \in X \setminus e_2$$
\end{cases}$$ $\Rightarrow |f| \leq g$ на $(X\setminus e_1)\setminus e_2$

\section{Простейшие свойства интеграла Лебега}
\subsection{Для определения (5)}
\begin{enumerate}
	\item $\int\limits_{\mathds{X}}f$ не зависит от представления $f$ как ступенчатой функции, то есть если $f$ реализуется как $f = \sum\limits_{k}(\lambda_k \cdot \chi_{E_k})$ и как $f = \sum\limits_{l}(\alpha_l \cdot \chi_{G_l})$, интегралы по этим функциям равны

	\emph{Доказательство:}

	Выпишем общее разбиение для этих двух разбиений

	Пусть $F_{ij} = E_i \cap G_j$

	Тогда $f = \sum\limits_{k}(\lambda_k \cdot \chi_{E_k}) = \sum\limits_{l}(\alpha_l \cdot \chi_{G_l}) = \sum\limits_{i, j}(\lambda_i (= \alpha_j) \cdot \chi_{F_{i, j}})$

	$\int f = \sum\limits_{i, j}(\lambda_i \cdot \mu F_{i, j}) = \sum\limits_i (\lambda_i \cdot \sum\limits_j (\mu F_{i, j})) = \sum\limits_i (\lambda_i \cdot \mu E_i) = \int f$ для первого разбиения

	Аналогично для второго разбиения получаем

	$\int f = \sum\limits_{i, j}(\lambda_i \cdot \mu F_{i, j}) = \sum\limits_j (\alpha_i \cdot \sum\limits_i (\mu F_{i, j})) = \sum\limits_j (\lambda_j \cdot \mu G_i) = \int f$ для второго разбиения, что и требовалось доказать

	\item $f, g$ -измеримые ступенчатые функции, $f \leqslant g$, тогда $\int\limits_{\mathds{X}} f \leqslant \int\limits_{\mathds{X}} g$

	\emph{Доказательство:}

	Пусть $f = \sum\limits_{k}(\lambda_k \cdot \chi_{E_k})$, $g = \sum\limits_{l}(\alpha_l \cdot \chi_{G_l})$

	Аналогично доказательству предыдущей теоремы, строим общее ступенчатое разбиение

	Пусть $F_{ij} = E_i \cap G_j$

	Тогда $\int f = \sum\limits_{i, j}(\lambda_i \cdot \mu F_{i, j}) \leqslant \sum\limits_j(\alpha_j \cdot \mu F_{i, j}) = \int g$, что и требовалось доказать
\end{enumerate}

\subsection{Для окончательного определения}

\begin{enumerate}
	\item Монотонность
	$f \leqslant g \Rightarrow \int\limits_{\mathds{X}} f \leqslant \int\limits_{\mathds{X}} g$

	\emph{Доказательство:}
	\begin{enumerate}
		\item $f, g \geqslant 0$, тогда доказательство тривиально (по свойствам супремума)
		\item $\int\limits_{\mathds{X}} f = \int\limits_{\mathds{X}} f^+ - \int\limits_{\mathds{X}} f^-$

		$\int\limits_{\mathds{X}} g = \int\limits_{\mathds{X}} g^+ - \int\limits_{\mathds{X}} g^-$

		Из того, что $\int\limits_{\mathds{X}} f^+ \leqslant \int\limits_{\mathds{X}} g^+$, а $\int\limits_{\mathds{X}} f^- \geqslant \int\limits_{\mathds{X}} g^-$ следует, что $\int\limits_{\mathds{X}} f \leqslant \int\limits_{\mathds{X}} g$
	\end{enumerate}

	\item
	$\int\limits_{\mathds{E}} 1 \cdot d \mu = \mu E$

	$\int\limits_{\mathds{E}} 0 \cdot d \mu = 0$

	Очевидно из определения интеграла ступенчатой функции

	\item $\mu E = 0, f $-измерима, тогда $\int\limits_{\mathds{E}}f = 0$, даже если $f = \infty$ на $\mathds{E}$

	\emph{Доказательство:}

	\begin{enumerate}
		\item $f $-ступенчатая $\Rightarrow$ ограниченная

		$f = \sum\limits_{k = 1}^{n}(\lambda_k \cdot \chi_{E_k})$, тогда $\int\limits_\mathds{E} f = \sum \lambda_k \cdot \mu (E \cap E_k)$

		Но $\mu (E \cap E_k) = 0$ (так как $\mu E = 0$), тогда $\int\limits_\mathds{E} f = 0$

		\item $f$ - измеримая, $f \geqslant 0$.

		$\int\limits_\mathds{E} f = \sup (\int\limits_\mathds{E} g)$, где $0 \leqslant g \leqslant f$, $g$ - ступенчатая

		Тогда $\int\limits_\mathds{E} f = \sup (0) = 0$

		\item
		$f$ - произвольная измеримая

		Тогда $\int\limits_\mathds{E} f = \int\limits_{\mathds{E}} f^+ - \int\limits_{\mathds{E}} f^- = 0 - 0 = 0$
	\end{enumerate}

	\item
	\begin{enumerate}
		\item
		$\int\limits_{\mathds{E}} -f = - \int\limits_{\mathds{E}} f$

		\item
		$\forall c \in \mathds{R}: \int\limits_{\mathds{E}} (c \cdot f) = c \cdot \int\limits_{\mathds{E}} f $
	\end{enumerate}

	\emph{Доказательство:}
	\begin{enumerate}
		\item
		$(-f)^+ = f^-$

		$(-f)^- = f^+$

		Тогда $\int\limits_{\mathds{E}} -f = \int\limits_{\mathds{E}} (-f)^+ - \int\limits_{\mathds{E}} (-f)^- = \int\limits_{\mathds{E}} f^- - \int\limits_{\mathds{E}} f^+ = -\int\limits_{\mathds{E}} f$

		\item

		Пусть $c > 0$. Если $c < 0$, то по предыдущему случаю можем рассматривать для $- c < 0$. Если $c = 0$, то по предыдущей теореме $\int\limits_{\mathds{E}} (0 \cdot f) = \int\limits_{\mathds{E}} 0 = 0 = 0 \cdot \int\limits_{\mathds{E}} f$

		\begin{enumerate}
			\item
			Пусть $f \geqslant 0$

			$\int\limits_{\mathds{E}} (c \cdot f) = \sup (\int\limits_{\mathds{E}} g)$, где $0 \leqslant g \leqslant c \cdot f$, $g$ - ступенчатая

			Пусть $g = c \cdot \widetilde{g}$, тогда $\int\limits_{\mathds{E}} (c \cdot f) = \sup (\int\limits_{\mathds{E}} (c \cdot \widetilde{g}))$, где $0 \leqslant c \cdot \widetilde{g} \leqslant c \cdot f$, $\widetilde{g}$ - ступенчатая

			Тогда $\int\limits_{\mathds{E}} (c \cdot f) = \sup (\int\limits_{\mathds{E}} (c \cdot \widetilde{g})) = \sup (c \cdot \int\limits_{\mathds{E}} \widetilde{g}) = c \cdot \sup (\int\limits_{\mathds{E}} \widetilde{g}) = c \cdot \int\limits_{\mathds{E}} f $

			\item Если $f$ - произвольная:

			$\int\limits_{\mathds{E}} (c \cdot f) = \int\limits_{\mathds{E}} (c \cdot f)^+ - \int\limits_{\mathds{E}} (c \cdot f)^- = \int\limits_{\mathds{E}} c \cdot f^+ - \int\limits_{\mathds{E}} c \cdot f^- = c \cdot \int\limits_{\mathds{E}} f^+ - c \cdot \int\limits_{\mathds{E}} f^- = c \cdot (\int\limits_{\mathds{E}} f^+ - \int\limits_{\mathds{E}} f^-) = c \cdot \int\limits_{\mathds{E}} f$
		\end{enumerate}
	\end{enumerate}

	\item Если существует $\int\limits_{\mathds{E}} f d\mu$, то $|\int\limits_{\mathds{E}} f| \leqslant \int\limits_{\mathds{E}} |f|$

	\emph{Доказательство:}

	$-|f| \leqslant f \leqslant |f|$

	$\int\limits_{\mathds{E}} -|f| \leqslant \int\limits_{\mathds{E}} f \leqslant \int\limits_{\mathds{E}} |f|$

	$-\int\limits_{\mathds{E}} |f| \leqslant \int\limits_{\mathds{E}} f \leqslant \int\limits_{\mathds{E}} |f|$

	Тогда $|\int\limits_{\mathds{E}} f| \leqslant \int\limits_{\mathds{E}} |f|$


	\item $f$ - измеримая на $\mathds{E}$, $\mu \mathds{E} < \infty$

	$a \leqslant f \leqslant b$, тогда $a \cdot \mu E \leqslant \int\limits_{\mathds{E}} f \leqslant b \cdot \mu E$

	\emph{Доказательство: }

	$a \leqslant f \leqslant b \Rightarrow \int\limits_{\mathds{E}}a \leqslant \int\limits_{\mathds{E}} f \leqslant \int\limits_{\mathds{E}} b$

	$a \cdot \int\limits_{\mathds{E}} 1 \leqslant \int\limits_{\mathds{E}} f \leqslant b \cdot \int\limits_{\mathds{E}} 1$

	$a \cdot \mu \mathds{E} \leqslant \int\limits_{\mathds{E}} f \leqslant b \cdot \mu \mathds{E}$

	\emph{Следствие:}

	Если $f$ - Измеримая и ограниченная на $\mathds{E}, \mu \mathds{E} < \infty$, тогда $f$ - суммируемая на $\mathds{E}$

	\item $f$ - суммируемая на $\mathds{E} \Rightarrow f$ почти везде конечная на $\mathds{E}$ (то есть $f \in \alpha^0(\mathds{E})$)

	\emph{Доказательство:}
	\begin{enumerate}
		\item Пусть $f \geqslant 0$

		Пусть $f = +\infty$ на $A$ и пусть $\mu A > 0$

		Тогда $\forall n \in \mathds{N}: f \geqslant n \cdot \chi_A$

		Тогда $\forall n \in \mathds{N}: \int\limits_{\mathds{E}} f \geqslant n \cdot \int\limits_{\mathds{E}} \chi_A = n \cdot \mu A \Rightarrow \int\limits_{\mathds{E}} f = + \infty$

		\item $f$ любого знака

		Распишем $f = f^+ - f^-$, по предыдущему пункту $f^+, f^-$ конечны почти везде $\Rightarrow f$ тоже конечно почти везде
	\end{enumerate}

\end{enumerate}

\section{Счетная аддитивность интеграла (по множеству)}
$(X,\mathds{A},\mu)$~--- пространство с мерой, $A = \bigsqcup\limits_{i=1}^{\infty}A_{i} -$ измеримы. $f: X \rightarrow \mathbb{\overline{R}} - $ изм., $f \geqslant 0$

\emph{Тогда:} ${\displaystyle \int\limits_{A}f = \sum\limits_{i=1}^{\infty} \int\limits_{A_{i}}f}$

\emph{Доказательство:}

\begin{enumerate}
	\item Для начала докажем это для ступенчатых функций. Пусть $f = \sum\limits_{k} (\lambda_k \cdot \chi_{E_k})$

 $\int\limits_{A}fd\mu = \sum\limits_{k} (\lambda_k \cdot \mu(E_k \cap A)) =
 \sum\limits_{k} (\lambda_k \cdot (\sum\limits_{i} \mu(E_k \cap A_i))) =
 \sum\limits_{i}(\sum\limits_{k}(\lambda_k \cdot \mu(E_k \cap A_i))) = \sum\limits_{i}(\int\limits_{A_i}f)$

	\item Докажем, что $\int\limits_{A}f \leqslant \sum\limits_{i} \int\limits_{A_{i}}f$

	\begin{enumerate}
		\item Рассмотрим $0 \leqslant g \leqslant f - $ ступенчатая. $\int\limits_{A}g = \sum\limits_{i} \int\limits_{A_i}g \leqslant \sum\limits_{i} \int\limits_{A_{i}}f$

		 \item Переходя к $sup$ получаем желаемое
	\end{enumerate}

	\item Теперь докажем, что $\int\limits_{A}f \geqslant \sum\limits_{i} \int\limits_{A_{i}}f$
	\begin{enumerate}
		\item $A = A_1\sqcup A_2$

		\begin{enumerate}
			\item Рассмотрим $g_1, g_2\ -$ ступенчатые такие, что $0 \leqslant g_i \leqslant f \cdot \chi_{A_i}$

			\item Рассмотрим их общее разбиение $E_k:\ g_i = \sum\limits_k (\lambda_k^i \cdot \chi_{E_k})$

			\item $g_1 + g_2\ - $ ступенчатая и $0 \leqslant g_1 + g_2 \leqslant f \cdot \chi_{A}$

			\item $\int\limits_{A_1}g_1 + \int\limits_{A_2}g_2 \stackrel{lemma}{=} \int\limits_{A}(g_1 + g_2) \stackrel{iii}{\leqslant} \int\limits_{A}f$

			\item Поочерёдно переходя к $sup$ по $g_1$ и $g_2$ получаем: $\int\limits_{A_1}f + \int\limits_{A_2}f \leqslant \int\limits_{A}f$
		\end{enumerate}

	\item $\forall n \in \mathbb{N}$, что $A = \bigsqcup\limits_{i=1}^{n}A_{i}$ будем последовательно отщеплять последнее множество по $(a)$

	\item $A = \bigsqcup\limits_{i = 1}^{\infty}A_{i}$
		\begin{enumerate}
			\item Фиксрируем $n \in \mathbb{N}$

			\item $A = (\bigsqcup\limits_{i=1}^{n}A_{i}) \sqcup B$, где $B = \bigsqcup\limits_{i=n+1}^{\infty}A_{i}$

			\item $\int\limits_{A}f \geqslant \sum\limits_{i=1}^{n} \int\limits_{A_i}f + \int\limits_{B}f \geqslant \sum\limits_{i=1}^{n} \int\limits_{A_i}f$

			\item Переходим к $lim$ по $n$
		\end{enumerate}

	\end{enumerate}

\end{enumerate}

\emph{Следсвие 1:}
$\ 0 \leqslant f \leqslant g$ - измeримы и  $A \subset B$ - измеримы $\Rightarrow \int\limits_{A}f \leqslant \int\limits_{B}g$

$\int\limits_{B}g \geqslant \int\limits_{B}f = \int\limits_{A}f + \int\limits_{B \setminus A}f \geqslant \int\limits_{A}f$

\bigskip

\emph{Следствие 2:}
$f$ - суммируема на $A \Rightarrow \int\limits_{A}f = \sum\limits_{i} \int\limits_{A_{i}}f$

Достаточно рассмотреть срезки $f^+$ и $f^-$

\bigskip

\emph{Следствие 3:}
$f \geqslant 0$ - изм. $\delta: \mathbb{A} \rightarrow \mathbb{\overline{R}}(A\longmapsto \int\limits_{A}fd\mu) \Rightarrow \delta$ - мера

\section{Теорема Леви}
$(X,\mathds{A},\mu),\ f_n \geqslant 0$ - изм.

$f_1(x) \leqslant ...\leqslant f_n(x) \leqslant f_{n+1}(x) \leqslant ...$ при почти всех $x$

$f(x) = \lim\limits_{n \rightarrow \infty}f_n(x)$ при почти всех $x$ (считаем, что при остальных $x: f \equiv 0$)
\\

\emph{Тогда:} $\lim\limits_{n \rightarrow \infty} \int\limits_{X}f_n(x)d\mu = \int\limits_{X}f(x)d\mu$
\\

\emph{Доказательство:}

$N.B. \int\limits_{X}f_n \leqslant \int\limits_{X}f_{n+1} \Rightarrow \exists \lim$

$f$ - измерима как предел последовательности измеримых функций

\begin{enumerate}
	\item $\leqslant$

	Очевидно $f_n \leqslant f$ при п.в $x \Rightarrow \int\limits_{X}f_n \leqslant \int\limits_{X}f$. Делаем предельный переход по $n$

	\item $\geqslant$
		\begin{enumerate}
			\item Логичная редукция: $\lim\limits_{n \rightarrow \infty} \int\limits_{X}f_n(x) \geqslant \int\limits_{x}g$, где $0 \leqslant g \leqslant f$ - ступенчатая

			\item Наглая редукция: $\forall c \in (0,1): \lim\int\limits_{X}f_n(x) \geqslant c \cdot \int\limits_{X}g$
				\begin{enumerate}
					\item $E_n = \{x\ |\ f_n(x) \geqslant c \cdot g\}$. Очевидно $E_1 \subset ... \subset E_n \subset E_{n + 1} \subset ...$

					\item $\bigcup\limits_{n=1}^{\infty}E_n = X$ т.к. $c < 1$

					\item $\int\limits_{X}f_n \geqslant \int\limits_{E_n}f_n \geqslant \int\limits_{E_n}g \Rightarrow \lim \int\limits_{X}f_n \geqslant c \cdot \lim \int\limits_{E_n}g = c \cdot \int\limits_{X}g$

					\item Последний знак равно обусловлен тем, что интеграл неотрицательной и измеримой функции по множеству - мера (см. следствие 3 предыдущей теоремы), и мы используем неперрывность меры снизу
				\end{enumerate}
		\end{enumerate}
\end{enumerate}

\section{Линейность интеграла Лебега}

$f, g \geqslant 0$, измеримые

Тогда $\int\limits_{\mathds{E}} (f + g) = \int\limits_{\mathds{E}} f + \int\limits_{\mathds{E}} g$

\emph{Доказательство:}

\begin{enumerate}
	\item Пусть $f, g$ - ступенчатые, тогда у них имеется общее разбиение

	$f = \sum\limits_{k}(\lambda_k \cdot \chi_{E_k})$

	$g = \sum\limits_{k}(\alpha_k \cdot \chi_{E_k})$

	$\int\limits_{\mathds{E}} (f + g) = \sum\limits_k (\lambda_k + \alpha_k) \cdot \mu E_k = \sum\limits_k \lambda_k \cdot \mu E_k + \sum\limits_k \alpha_k \cdot \mu E_k = \int\limits_{\mathds{E}} f + \int\limits_{\mathds{E}} g$, что и требовалось доказать

	\item $f, g \geqslant 0$, измеримые

	Тогда $\exists h_n: 0 \leqslant h_n \leqslant h_{n + 1} \leqslant f$, $h_n$ ступенчатые

	$\exists \widetilde{h_n}: 0 \leqslant \widetilde{h_n} \leqslant \widetilde{h_{n + 1}} \leqslant g$, $\widetilde{h_n}$ ступенчатые

	$\lim\limits_{n \rightarrow +\infty} h_n = f$

	$\lim\limits_{n \rightarrow +\infty} \widetilde{h_n} = g$

	$\int\limits_{\mathds{E}} (h_n + \widetilde{h_n}) = \int\limits_{\mathds{E}} h_n + \int\limits_{\mathds{E}} \widetilde{h_n}$

	$\int\limits_{\mathds{E}} (h_n + \widetilde{h_n}) \rightarrow \int\limits_{\mathds{E}} (f + g)$

	$\int\limits_{\mathds{E}} h_n \rightarrow \int\limits_{\mathds{E}} f$

	$\int\limits_{\mathds{E}} \widetilde{h_n} \rightarrow \int\limits_{\mathds{E}} g$

	Тогда $\int\limits_{\mathds{E}} (f + g) = \int\limits_{\mathds{E}} f + \int\limits_{\mathds{E}} g$, что и требовалось доказать

	\item
	Если $f, g$ - любые измеримые, распишем обе через срезки и докажем для них
\end{enumerate}

\section{Теорема об интегрировании плоложительных рядов}
$u_n(x) \geq 0$ \textit{почти всюду} на $\mathds{E}$, тогда
$\int\limits_{\mathds{E}} (\sum\limits_{n=1}^{+\infty}u_n(x))d\mu(x) =
\sum\limits_{n=1}^{+\infty} \int\limits_{\mathds{E}} u_n(x)d\mu(x)$

\underline{Доказательство:} \\
$S_N(x) = \sum\limits_{n=1}^{N}u_n(x)$; $S(x) = \sum\limits_{n=1}^{+\infty}u_n(x)$
\begin{enumerate}
	\item $S_N$ - возрастает к $ S $ при почти всех x $\xRightarrow{\text{Т. Леви}} \int\limits_{\mathds{E}}S_N \xrightarrow[N \rightarrow +\infty] {}  \int\limits_{\mathds{E}}S = \int\limits_{\mathds{E}}\sum\limits_{n=1}^{+\infty}u_n(x)$
	\item С другой стороны $\int\limits_{\mathds{E}}S_N = \int\limits_{\mathds{E}} \sum\limits_{n=1}^{N}u_n = \sum\limits_{n=1}^{N}\int\limits_{\mathds{E}}u_n(x)d\mu \xrightarrow[N \rightarrow +\infty] {} \sum\limits_{n=1}^{+\infty}\int\limits_{\mathds{E}}u_n(x)d\mu$
	\item Найденные пределы совпадают в силу единственности предела последовательности, что и требовалось доказать.
\end{enumerate}

\section{Теорема о произведении мер}
$<\mathds{X}, \alpha, \mu>$, $<\mathds{Y}, \beta, \nu>$ - пространства с мерой\\
$\alpha \times \beta = \{A\times B \subset \mathds{X} \times \mathds{Y} : A \in \alpha, B \in \beta \}$ \\
$m_0(A \times B) = \mu A \cdot \nu B$
\\\\
Тогда:
\begin{enumerate}
	\item $m_0$ - мера на полукольце $\alpha \times \beta$
	\item $\mu$, $\nu$ - $\sigma$-конечны $\Rightarrow$ $m_0$ - $\sigma$-конечна\\
\end{enumerate}

\underline{Доказательство:} \\
\begin{enumerate}
	\item Неотрицательность $m_0$ очевидна. Необходимо доказать счетную аддитивность\\
	Пусть $P = \bigsqcup\limits_{i=1}^{\infty}P_{k}$, где $P \in \alpha \times \beta$ \\
	$P = A \times B$; \ $P_k = A_k \times B_k$ \\
	Заметим, что:
	\begin{itemize}
		\item $\chi_P(x, y) = \sum\chi_{P_k}(x, y)$, в силу дизъюнктности $P_k$ ((x, y) входит максимум в одно множество из всех $P_k$)
		\item $\chi_{A \times B}(x, y) = \chi_A(x) \cdot \chi_B(y)$, так как (x, y) $\in A\times B \Leftrightarrow x \in A$ И $y \in B$
	\end{itemize}
Воспользовавшись вышесказанным получим:\\
$\chi_{P}(x, y) = \chi_{A\times B}(x, y) = \chi_A(x) \cdot \chi_B(y)$\\
$\chi_{P}(x, y) = \sum\chi_{P_k}(x, y) = \sum\chi_{A_k \times B_k}(x, y) = \sum\chi_{A_k}(x) \cdot \chi{B_k}(y)$

Имеем следующее равенство:\\
$\chi_A(x) \cdot \chi_B(y) = \sum\chi_{A_k}(x) \cdot \chi{B_k}(y)$

Проинтегрируем его по мере $\mu$ по x, затем по мере $\nu$ по y, получим:

$\mu A \cdot \nu B = \sum \mu A_k \cdot \nu B_k$, то есть $m_0(P) = \sum m_0(P_k)$, что и требовалось доказать.
\item $\mu$, $\nu$ - $\sigma$-конечны $\Rightarrow$
$X = \bigcup\limits_{k=1}^{\infty} A_k$, где $\mu A_k < +\infty$;\
$Y = \bigcup\limits_{k=1}^{\infty} B_k$, где $\nu B_k < +\infty$

$X \times Y = \bigcup\limits_{i, j} (A_i \times B_j)$

$m_0(A_i \times B_j) = \mu A_i \cdot \nu B_j < +\infty$, так как $\mu A_i < +\infty$ и $\nu B_j < +\infty$

все $(A_i \times B_j) \in \alpha \times \beta$ по определению

Что и требовалось доказать.
\end{enumerate}

\section{Абсолютная непрерывность интеграла}
$<\mathds{X}, \alpha, \mu>$ - пространство с мерой\\
$f : X \to \overline{\mathds{R}}$ - суммируема\\\\
Тогда $\forall \epsilon > 0 ~ \exists \delta > 0 : ~ \forall E - \text{измеримое} ~ \mu E < \delta ~ |\int\limits_{E}f d\mu| < \epsilon$

\underline{Доказательство:} \\
$X_n := X(|f| \geq n)$\\
$X_n \subset X_{n+1} \subset ...$\\
$\mu(\cap X_n) = 0$, т.к. f -- суммируема
\begin{enumerate}
	\item Мера : ($A \mapsto \int\limits_{A}|f|$) непрерывна сверху, т.е.
	$\forall ~ \epsilon ~ \exists ~ n_{\epsilon} ~ \int\limits_{X_{n_{\epsilon}}}|f| < \epsilon / 2$
	\item Зафиксируем $\epsilon$ в доказываемом утверждении, возьмем $\delta := \frac{\epsilon / 2}{n_{\epsilon}}$
	\item $|\int\limits_{E}f d\mu| \leq \int\limits_{E} |f| = \int\limits_{E \cap X_{n_{\epsilon}}}|f| ~+~ \int\limits_{E \cap X_{n_{\epsilon}}^c}|f| \overset{*}{\leq} \int\limits_{X_{n_{\epsilon}}}|f| ~+~ n_{\epsilon} \cdot \mu (E \cap X_{n_{\epsilon}}^c) \overset{**}{<} \epsilon / 2 + n_\epsilon \cdot \mu E < \epsilon / 2 + n_\epsilon \cdot \frac{\epsilon / 2}{n_{\epsilon}} < \epsilon$ \\
	* - В первом слагаемом увеличили множество, во втором посмотрели на определние $X_n$, взяли дополнение, воспользовались 6-м простейшим свойством интеграла\\
	** - Воспользовались непрерывностью сверху
\end{enumerate}
	\subsection{Следствие}
	f - суммируема\\
	$e_n$ - измеримые множества\\
	
	$\mu e_n \rightarrow 0 \Rightarrow \int\limits_{e_n}f \rightarrow 0$
\section{Теорема Лебега о мажорированной сходимости для случая сходимости по мере.}

\begin{flushleft}

$<\mathds{X}, \mathds{A}, \mu>$ -- пространство с мерой,

$f_n, f$ -- измеримы,

$f_n\stackrel{\mu}{\Rightarrow}f$ (сходится по мере),

$\exists g : \mathds{X} \rightarrow \overline{\mathds{R}}$ такая, что:
\begin{itemize}
\item
$\forall n$,  для <<почти всеx>> $x$ ~ $|f_n(x)| \leq g(x)$ ($g$ -- называется мажорантой)
\item
$g$ - суммируемая
\end{itemize}

\emph{\textbf{Тогда:}}
\begin{itemize}
    \item $f_n, f$ -- суммируемы
    \item $\int\limits_{\mathds{X}} |f_n - f| d\mu \rightarrow 0$
    \item $\int\limits_{\mathds{X}} f_n \rightarrow \int\limits_{\mathds{X}} f$ (<<уж тем более>>)
\end{itemize}

\emph{Доказательство:}

\begin{enumerate}
	\item $f_n$ -- суммируема, так как существует мажоранта $g$
	\item $f$ -- суммируема по теореме Рисса ($ f_n{_k} \rightarrow f $ почти везде, $ |f_n{_k}| \leq g$, тогда $|f| \leq g$ почти везде)
	\item <<уж тем более>>:

	$ |\int\limits_{\mathbb{X}} f_n - \int\limits_{\mathbb{X}} f| \leq \int\limits_{\mathbb{X}} |f_n - f| $

	Допустим, что $\int\limits_{\mathds{X}} |f_n - f| d\mu \rightarrow 0$ уже доказано.

	Тогда <<уж тем более>> очевидно.

	\item Докажем основное утверждение:

	Разберем два случая:
	\begin{enumerate}
		\item $ \mu \mathbb{X} < \infty $
		Фиксируем $ \epsilon \ge 0 $ ~ $ X_n := X(|f_n - f| \geq \epsilon) $

		$ \mu X \rightarrow 0 $ (так как $ f_n \Rightarrow f $)

		$ \int\limits_{\mathbb{X}} |f_n - f| =
		\int\limits_{X_n} |f_n - f| + \int\limits_{X_n^c} |f_n - f| \leq
		\int\limits_{X_n} 2g + \int\limits_{X_n^c} \epsilon < \epsilon + \epsilon \mu \mathbb{X} $ (прим. $ \int\limits_{X_n} 2g \rightarrow 0 $ по след. к т. об абс. сходимости )

		\item $ \mu \mathbb{X} = \infty $

		Докажем <<Антиабсолютную непрерывность>> для $ g $:

		$ \forall \epsilon ~ \exists A \subset \mathbb{X} \mid \mu A$ - конеч.
		$ \int\limits_{X\backslash A} g < \epsilon $

		\textit{доказательство:}

		$ \int\limits_{\mathbb{X}} = \sup ( \int\limits_{\mathbb{X}} g_k \mid 0 \leq g_k \leq g) $ ($ g_k $ -- ступен.)

		$ \exists g_n ~ \int\limits_{\mathbb{X}} g - \int\limits_{\mathbb{X}} g_n < \epsilon $

		$ A:= supp\ g_n $ ($ supp\ f := $ замыкание \{$ x \mid f(x) \neq 0$ \})

		$ A =  \bigcup\limits_{k \mid \alpha_k \neq 0} E_k $

		$ g = \sum\limits_{kon} \alpha_k \mathcal{X}_{E_k} $ ($ X = \bigsqcup E_k $)

		$ \int\limits_{\mathbb{X}} g_n  = \sum \alpha_k \mu E_k  < +\infty $ ($ \mu A $ - конеч.)

		$ \int\limits_{\mathbb{X\backslash A}} g =
		\int\limits_{\mathbb{X\backslash A}} g - g_n \leq
		\int\limits_{\mathbb{X}} g - g_n < \epsilon $

		Теперь докажем основное утверждение:

		 $ \int\limits_{\mathbb{X}} |f_n - f| =
		 \int\limits_{\mathbb{A}} |f_n - f| + \int\limits_{\mathbb{X\backslash A}} |f_n - f| \leq
		 \int\limits_{\mathbb{A}} |f_n - f| + 2\epsilon < 3 \epsilon $
		 ($  \int\limits_{\mathbb{A}} |f_n - f| \rightarrow 0$  по п. (a))
	\end{enumerate}
\end{enumerate}
\end{flushleft}
\section{Теорема Лебега о мажорированной сходимости для случая сходимости почти везде.}
$<\mathds{X}, \mathds{A}, \mu>$ -- пространство с мерой,

$f_n, f$ -- измеримы,

$f_n\stackrel{\mu}{\Rightarrow}f$ \textbf{почти везде},

$\exists g \mid \mathds{X} \rightarrow \overline{\mathds{R}}$ такая, что:
\begin{itemize}
\item
$\forall n$,  для <<почти всеx>>$x ~ |f_n(x)| \leq g(x)$ ($g$ -- называется мажорантой)
\item
$g$ - суммируемая
\end{itemize}

\emph{\textbf{Тогда:}}
\begin{itemize}
    \item $f_n, f$ -- суммируемы
    \item $\int\limits_{\mathds{X}} |f_n - f| d\mu \rightarrow 0$
    \item $\int\limits_{\mathds{X}} f_n \rightarrow \int\limits_{\mathds{X}} f$ (<<уж тем более>>)
\end{itemize}

\emph{Доказательство:}

\begin{enumerate}
	\item <<уж тем более>> см. пред. теорему.

	\item Докажем основное утверждение:

	$ h_n(x) := \sup(|f_n - f|, |f_{n+1} - f|, \dots) $

	Заметим, что при фикс. $ x $ выпол. $ 0 \leq h_n \leq 2g $ почти везде

	$ \lim\limits_{n \rightarrow +\infty} h_n =
	\varlimsup\limits_{n \rightarrow +\infty} |f_n - f| = 0 $ почти везде

	$ 2g - h_n \uparrow , ~ 2g - h_n \rightarrow 2g$ почти везде

	$ \int\limits_{\mathds{X}}(2g - h_n) d\mu \rightarrow \int\limits_{\mathds{X}} 2g $
	(по т. Леви)

	$ \int\limits_{\mathds{X}} 2g - \int\limits_{\mathds{X}} h \Rightarrow \int\limits_{\mathds{X}} h_n \rightarrow 0 $

	$ \int\limits_{\mathds{X}} |f_n - f| \leq \int\limits_{\mathds{X}} h_n \rightarrow 0 $
\end{enumerate}

\section{Теорема Фату. Следствия.}
$<\mathbb{X}, \mathbb{A}, \mu>$ -- пространство с мерой

$f_n, f$ -- измеримы,

$f_n \geq 0$

$f_n\stackrel{\mu}{\Rightarrow}f$ <<почти везде>>,

$\exists C > 0 ~ \forall n ~ \int\limits_{\mathbb{X}} f_n d\mu \leq C$

\textbf{\emph{Тогда:}}
\begin{itemize}
\item $\int\limits_{\mathbb{X}}f \leq C$
\end{itemize}

\emph{Доказательство:}

$ g_n := \inf(f_n, f_{n+1}, \dots ) $ ~ ($ g_n \leq g_{n+1} \leq \dots $)

$ \lim g_n = \varliminf(f_n) = $ \textit{почти везде} $ = \lim f_n  = f$ ($ g_n \rightarrow f $ почти везде)


$ \int\limits_{\mathbb{X}} g_n \leq \int\limits_{\mathbb{X}} f_n \leq C $

$ \int\limits_{\mathbb{X}} f = $ \textit{по т. Леви} $ = \lim \int\limits_{\mathbb{X}} g_n \leq C$

\subsection{Следствие 1}

$ f_n, f \geq 0$ -- измер.

$ f_n \stackrel{\mu}{\Rightarrow} f$

$ Пусть \exists C ~ \forall n  \int\limits_{\mathbb{X}} f_n \leq C $

\emph{Тогда:}
\begin{itemize}
	\item $ \int\limits_{\mathbb{X}}  f \leq C $
\end{itemize}

\emph{Доказательство:}

$ \exists f_{n_k} \rightarrow f $ почти везде

\subsection{Следствие 2}

$ f_n \geq 0 $ -- измер.

\emph{Тогда:}

\begin{itemize}
	\item $ \int\limits_{\mathbb{X}} \underline{lim}( f_n ) \geq \underline{lim}( \int\limits_{\mathbb{X}} f_n ) $
\end{itemize}

\emph{Доказательство:}

$ \exists n_k \mid \int\limits_{\mathbb{X}} f_{n_k} \underrightarrow{k \rightarrow +\infty}
\varliminf\limits_{n \rightarrow +\infty} \int\limits_{\mathbb{X}} f_n $

Рассмотрим $ g_{n_k} $ такое, что $ g_{n_k} \uparrow $ и $ g_{n_k} \rightarrow \varliminf f $

Применяем теорему Леви к нер-ву $ \int\limits_{\mathbb{X}} g_{n_k} \leq \int\limits_{\mathbb{X}} f_{n_k}$

$ \int\limits_{\mathbb{X}} \varliminf f \leq \varliminf \int\limits_{\mathbb{X}} f_n $

\end{document}
